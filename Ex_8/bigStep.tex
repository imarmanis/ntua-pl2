\documentclass[a4paper,oneside, 12pt, fleqn]{article}
\usepackage[margin=0.7in]{geometry}

\usepackage[cm-default]{fontspec}
\usepackage{xunicode}
\usepackage{xltxtra}

\usepackage{mathtools}
\allowdisplaybreaks

\usepackage{floatflt,amsmath,amssymb}
\usepackage[ligature, inference]{semantic}
\usepackage{syntax}


\usepackage{xgreek}

\setmainfont[Mapping=tex-text]{CMU Serif}



\title{
	\textbf{Γλώσσες Προγραμματισμού II}\\
	Άσκηση 8\\
	Συστήματα τύπων\\
	Σημασιολογία μεγάλων βημάτων
 }
\author{
	Μαρμάνης Ιάσων, 03114088
}

\date{\today}

\begin{document}

\maketitle


%\mathlig{->}{\rightarrow}
%\mathlig{|-}{\vdash}
%\mathlig{=>}{\Rightarrow}
%\mathligson

\paragraph*{Ορισμός γλώσσας}
\begin{align*}
	s &\Coloneqq (e, m)\\
	\\
	e &\Coloneqq n \mid true \mid false
	\mid -e \mid \neg e \mid e_1 + e_2 
	\mid e_1 \wedge e_2 \mid e_1 < e_2
	\mid if\; e\; then\; e_1\; else\; e_2\\
	&\mid x \mid \lambda x.\; e \mid e_1 e_2\\
	&\mid ref\; e \mid \; ! e \mid e_1 \Coloneqq e_2 \mid loc_i \mid unit\\
	\\
	u &\Coloneqq n \mid true \mid false\\
	&\mid \lambda x.\; e \\
	&\mid loc_i \mid unit
\end{align*}

\paragraph*{Σημασιολογία}
\[
\inference
{
	(e_1, m) \Downarrow (u_1, m') 
	\quad (e_2, m')  \Downarrow (u_2, m'')
	\quad \left([\![\circ]\!](u_1, u_2), m''\right)
		\Downarrow (u, m^{'''})
}
{
	(e_1 \circ e_2, m) \Downarrow 
	(u, m^{'''})
}
\]
\[
\inference
{
	(e, m) \Downarrow (u, m')
	\quad ([\![\diamond]\!](u), m') \Downarrow (u', m'')
}
{ 
	(\diamond e, m) \Downarrow (u', m'')
}
\]

\[
\inference
{(e, m) \Downarrow (true, m') \quad (e_1, m')
	 \Downarrow (u, m'')}
{(if\; e \; then\; e_1\; else\; e_2, m) \Downarrow (u, m'')}
\quad\quad
\inference
{(e, m) \Downarrow (false, m') \quad (e_2, m')
	\Downarrow (u, m'')}
{(if\; e \; then\; e_1\; else\; e_2, m) \Downarrow (u, m'')}
\]

\[
\inference
{(e_1, m) \Downarrow (\lambda x.\; e, m') 
	\quad (e_2, m') \Downarrow (u, m'')
	\quad (e[x \coloneqq u], m'') \Downarrow (u', m^{'''})
}
{(e_1 e_2, m) \Downarrow 
	\left(u, m^{'''}\right)}
\]

\[
\inference
{
	(e, m) \Downarrow (loc_i, m')
	\quad m'(i) = u
}
{(!e, m) \Downarrow (u,m')}
\qquad
\inference
{
	(e_1, m) \Downarrow (loc_i, m')
	\quad (e_2, m') \Downarrow (u, m'')
}
{
	(e_1 \coloneqq e_2, m) \Downarrow \left(unit, 
		m'' \left\{i \mapsto u \right\} \right)}
\]


\[
\inference
{
	(e, m) \Downarrow (u, m')
	\qquad j = max\left(dom\left(m'\right) \right) + 1
}
{(ref\;e, m) \Downarrow (loc_j, 
	m')\left\{j \mapsto u \right\}}
\]

\paragraph*{Σύστημα τύπων}
\begin{align*}
\tau &\Coloneqq Int \mid Bool\\
&\mid \tau_1 \rightarrow \tau_2\\
&\mid Ref\; \tau \mid Unit \\
\end{align*}
\[
\inference
{
	\Gamma; M \vdash e_1 : Int \quad \Gamma; M \vdash e_2 : Int
}
{
	\Gamma; M \vdash e_1 + e_2 : Int
}
\qquad
\inference
{
	\Gamma; M \vdash e : Int
}
{ 
	\Gamma; M \vdash -e : Int
}
\qquad
\inference
{
	\Gamma; M \vdash e : Bool
}
{ 
	\Gamma; M \vdash \neg e : Bool
}
\]

\[
\inference
{
	\Gamma; M \vdash e_1 : Bool \quad \Gamma; M \vdash e_2 : Bool
}
{
	\Gamma; M \vdash e_1 \wedge e_2 : Bool
}
\qquad
\inference
{
	\Gamma; M \vdash e_1 : Int \quad \Gamma; M \vdash e_2 : Int
}
{
	\Gamma; M \vdash e_1 < e_2 : Bool
}
\]

\[
\inference
{
	\Gamma; M \vdash e_1 : \tau \quad \Gamma; M \vdash e_2 : \tau
}
{
	\Gamma; M \vdash if\; e \; then\; e_1\; else\; e_2 : \tau
}
\qquad
\inference
{
	(x, \tau) \in \Gamma
}
{
	\Gamma; M \vdash x : \tau
}
\]

\[
\inference
{
	\Gamma, x:\tau\; ; M \vdash	e : \tau^{'}
}
{
	\Gamma; M \vdash \lambda x.\; e : \tau \rightarrow \tau^{'}
}
\qquad
\inference
{
	\Gamma; M \vdash e_1 : \tau \rightarrow \tau^{'}
	\quad \Gamma; M \vdash e_2 : \tau
}
{
	\Gamma; M \vdash e_1 e_2 : \tau^{'}
}
\]

\[
\inference
{
	\Gamma; M \vdash e : \tau
}
{
	\Gamma; M \vdash ref\; e : Ref\; \tau
}
\qquad
\inference
{
	\Gamma; M \vdash e : Ref\;\tau
}
{
	\Gamma; M \vdash \; !e : \tau
}
\]


\[
\inference
{
	\Gamma; M \vdash e_1 : Ref\; \tau
	\quad \Gamma; M \vdash e_2 : \tau
}
{
	\Gamma; M \vdash e_1 \coloneqq e_2 : Unit
}
\qquad
\inference
{
	M(i) = \tau
}
{
	\Gamma; M \vdash loc_i : Ref\;\tau
}
\]

\paragraph*{Θεώρημα ασφάλειας} Πρόοδος + Διατήρηση
\subparagraph*{Πρόοδος}
Άν $ \varnothing; M \vdash e : \tau$ τότε είτε $e$ τιμή είτε 
για κάθε $m$, τέτοιο ώστε $\varnothing \vdash m : M$, υπάρχει $(u,m')$  τέτοιο ώστε $(e,m) \Downarrow (u,m')$
\subparagraph*{Διατήρηση}
Αν $\Gamma; M \vdash e : \tau,\; \Gamma \vdash m : M$ και $(e,m) \Downarrow (u,m')$
τότε υπάρχει $M'$ τέτοιο ώστε 
$M\subseteq M',\; \Gamma \vdash m' : M'$ και $\Gamma ; M' \vdash u : \tau$

\paragraph*{Σύγκριση}
Με την σημασιολογία μεγάλων βημάτων χρειαζόμαστε λιγότερους κανόνες και αποφεύγουμε κάποια βήματα αλλά δεν μπορούμε να μελετήσουμε κολλημένες καταστάσεις. 

Με την σημασιολογία μικρών βημάτων έχουμε μεγαλύτερη λεπτομέρεια, η οποία όμως μπορεί να μην χρειάζεται, αφού εκφράζει μία-μία τις καταστάσεις από τις οποίες περνάει η αφηρημένη μηχανή.

Γενικά με την πρώτη κερδίζουμε απλότητα αλλά χάνουμε εκφραστικότητα.
Επίσης περιγράφει πιο άμεσα τον τρόπο που θα υλοποιούσε κάποιος την αφηρημένη μηχανή (διερμηνέα) για μία γλώσσα.
\end{document}